%%%%%%%%%%%%%%%%%%%%%%%%%%%%%%%%%%%%%%%%%
% Medium Length Professional CV
% LaTeX Template
% Version 2.0 (8/5/13)
%
% This template has been downloaded from:
% http://www.LaTeXTemplates.com
%
% Original author:
% Rishi Shah
%
% Important note:
% This template requires the resume.cls file to be in the same directory as the
% .tex file. The resume.cls file provides the resume style used for structuring the
% document.
%
%%%%%%%%%%%%%%%%%%%%%%%%%%%%%%%%%%%%%%%%%

%----------------------------------------------------------------------------------------
%	PACKAGES AND OTHER DOCUMENT CONFIGURATIONS
%----------------------------------------------------------------------------------------

\documentclass[UTF8]{cv} % Use the custom resume.cls style
%\usepackage{fontspec}
\usepackage[left=0.75in,top=0.6in,right=0.75in,bottom=0.6in]{geometry} % Document margins
%\usepackage{hyperref}
\usepackage[colorlinks,linkcolor=red,anchorcolor=blue,citecolor=green]{hyperref}
\usepackage{etaremune}
\usepackage[T1]{fontenc}
\usepackage{mathptmx}
%\usepackage{ctex}
\newcommand{\tab}[1]{\hspace{.2667\textwidth}\rlap{#1}}
\newcommand{\itab}[1]{\hspace{0em}\rlap{#1}}
\name{RUIHUI LI}  % Your name

%\address{Email:\href{mailto:ruihuili.lee@gmail.com}{ruihuili.lee@gmail.com}  \\ Homepage:\href{https://liruihui.github.io/}{https://liruihui.github.io/}
\address{Email: \href{mailto:ruihuili.lee@gmail.com}{ruihuili.lee@gmail.com} \\ Homepage: \href{https://liruihui.github.io/}{https://liruihui.github.io/}}
\address{Address: Rm 902, SHB, The Chinese University of Hong Kong, Shatin, Hong Kong.}
% Your address
%\address{123 Pleasant Lane \\ City, State 12345} % Your secondary addess (optional)
\address{Contact Number: (+86) 14715499208} % Your phone number and email

\begin{document}

%----------------------------------------------------------------------------------------
%	EDUCATION SECTION
%----------------------------------------------------------------------------------------

\begin{rSection}{Education}

{\bf Ph.D. in Computer Science} \hfill {08/2017 - 07/2021 (Expected)}
\\ The Chinese University of Hong Kong (CUHK), Hong Kong, China
\\ Advisor: Prof. Chi-Wing Fu\\
\\{\bf B.E \& M. Sc. in Computer Science and Technology} \hfill {09/2010 - 07/2017}
\\ Hunan University (HNU), Changsha, China 
%\\ GPA: 3.6/4.0\\

%\\{\bf B.E in Computer Science and Technology } \hfill {09/2010 - 07/2014}
%\\ Hunan University (HNU), China
%\\ GPA: 3.5/4.0\\
%Minor in Linguistics \smallskip \\
%Member of Eta Kappa Nu \\
%Member of Upsilon Pi Epsilon \\


\end{rSection}

\begin{rSection}{RESEARCH INTERESTS}
3D Vision, Point Cloud Processing, Computer Graphics and Deep Learning.
\end{rSection}


\begin{rSection}{PUBLICATIONS}
\hspace{-7mm} Conference:

\begin{etaremune}[itemindent=0.00em]
    \renewcommand\labelenumi{[C\theenumi]}
    \item  \emph{PointAugment: an Auto-Augmentation Framework for Point Cloud Classification}
\\    \textbf{Ruihui Li}, Xianzhi Li, Pheng-Ann Heng, and Chi-Wing Fu.
\\    IEEE Conference on Computer Vision and Pattern Recognition (\textbf{CVPR}, CCF-A), 2020. (\textbf{Oral})
    \item  \emph{PU-GAN: a Point Cloud Upsampling Adversarial Network}
\\    \textbf{Ruihui Li}, Xianzhi Li, Chi-Wing Fu, Daniel Cohen-Or, and Pheng-Ann Heng.
\\    IEEE International Conference on Computer Vision (\textbf{ICCV}, CCF-A, Citations: 23), 2019.
 
\end{etaremune}


\hspace{-7mm} Journal:

\begin{etaremune}[itemindent=0.00em]
    \renewcommand\labelenumi{[J\theenumi]}
    \item  \emph{DNF-Net: a Deep Normal Filtering Network for Mesh Denoising}
\\    Xianzhi Li, \textbf{Ruihui Li}, Lei Zhu, Chi-Wing Fu and Pheng-Ann Heng.
\\    IEEE Transactions on Visualization and Computer Graphics  (\textbf{TVCG}, CCF-A), 2020.
    \item  \emph{Aggregating Complementary Boundary Contrast with Smoothing for Salient Region Detection}
\\    \textbf{Ruihui Li}, Jianrui Cai, Hanling Zhang and Taihong Wang.
\\    The Visual Computer (\textbf{TVC}, CCF-C), 2017.
    \item \emph{Enhancing Augmented VR Interaction via Egocentric Scene Analysis}
\\  Yang Tian, Chi-Wing Fu, Shengdong Zhao, \textbf{Ruihui Li}, Xiao Tang, Xiaowei Hu, and Pheng-Ann Heng.
\\ACM on Interactive, Mobile, Wearable and Ubiquitous Technologies (\textbf{IMWUT}, CCF-A), 2019.
\end{etaremune}

%\hspace{-7mm} Journal Papers

%\begin{etaremune}[itemindent=0.00em]
%    \renewcommand\labelenumi{[J\theenumi]}
%    \item  \emph{Aggregating complementary boundary contrast with smoothing for salient region detection}
%\\    \textbf{Ruihui Li}, Jianrui Cai, Hanling Zhang and Taihong Wang
%\\   The Visual Computer (\textbf{TVC}), 2017
%\end{etaremune}

\hspace{-7mm} Manuscripts under review:
\begin{etaremune}[itemindent=0.00em]
    \renewcommand\labelenumi{[M\theenumi]}
    \item  \emph{Non-Local Part-Aware Point Cloud Denoising}
\\    Chao Huang$^*$, \textbf{Ruihui Li}$^*$, Xianzhi Li, Pheng-Ann Heng, and Chi-Wing Fu. ($^*$joint 1st authors)
\\    IEEE Transactions on Visualization and Computer Graphics  (\textbf{TVCG}, CCF-A), 2020.
    \item  \emph{A Rotation-Invariant Framework for Deep Point Cloud Analysis}
\\    Xianzhi Li, \textbf{Ruihui Li}, Chi-Wing Fu, Guangyong Chen, Daniel Cohen-Or, and Pheng-Ann Heng.
\\    IEEE Transactions on Visualization and Computer Graphics  (\textbf{TVCG}, CCF-A), 2020.
\end{etaremune}

\vspace{2cm}

\end{rSection}

\begin{rSection}{AWARDS AND HONORS}

    {National Scholarships (the \textbf{highest} scholarship for graduate students in China)} \hfill {2016} \\
    {Second Prize in Intel Cup Undergraduate Electronic Design Contest (Advisor: Xu Cheng)} \hfill {2014}\\
    {Gold Award of Pan-Pearl-River-Delta University IT Project Competition in China} \hfill {2014}\\
    %{Second Prize in Mathematical modeling contest } \hfill {2013}\\
    %{Second Prize in Netjava Robot Programming Contest} \hfill {2012}\\
    {First Prize Undergraduate Scholarship}\hfill {2011\&2012\&2013}\\
    {Award of Pacemaker to Merit Student} \hfill {2011}
    %{Award of Excellent Student Cadre} \hfill {2011}
    %\\ Ph.D. in Computer Science and Engineering
    %\\ Advisor: Prof. Chi-Wing Fu\\
\end{rSection}
%--------------------------------------------------------------------------------
%    Projects And Seminars
%-----------------------------------------------------------------------------------------------
%----------------------------------------------------------------------------------------
%	TECHNICAL STRENGTHS SECTION
%----------------------------------------------------------------------------------------

\begin{rSection}{Technical Skills}

\begin{tabular}{ @{} >{\bfseries}l @{\hspace{6ex}} l }
Languages \ & Python, C/C++, Java, Matlab \\
Toolkits \ & OpenGL, TensorFlow, Pytorch \\
\end{tabular}

\end{rSection}

%----------------------------------------------------------------------------------------
%	WORK EXPERIENCE SECTION
%----------------------------------------------------------------------------------------

\begin{rSection}{TEACHING Assistant}\itemsep -3pt
  \item  {CSCI 5210 Advanced Topics in Computer Graphics and Visualization} \hfill {Spring 2020}
  \item  {CSCI 3260 Principles of Computer Graphics} \hfill {Fall 2018\&2019}
  \item  {CSCI 3180 Principles of Programming Language} \hfill {Spring 2019}
  \item {ENGG 1110J Problem Solving by Programming} \hfill {Spring 2018}
  \item {CSCI 1130 Introduction to Computing Using Java} \hfill {Fall 2017}
\end{rSection}

%\begin{rSection}{PROFESSIONAL ACTIVITIES}

%Journal Reviews
%\begin{itemize}
%\vspace{-0.5em}
%\renewcommand\labelitemi{$\diamond$}
%\setlength\itemsep{-0.5em}
%  \item as
%  \item as
%  \item as
%\end{itemize}

%Teaching Assistant
%\begin{itemize}
%\vspace{-0.5em}
%\renewcommand\labelitemi{$\diamond$}
%\setlength\itemsep{-0.5em}
%  \item  {CSCI 3260 Principles of Computer Graphics} \hfill {Fall 2018\&2019}
%  \item  {CSCI3180 Principles of Programming Lang} \hfill {Spring 2019}
%  \item {ENGG1110J Problem Solving by Programming} \hfill {Spring 2018}
%  \item {CSCI1130 Introduction to Computing Using Java} \hfill {Fall 2017}
%\end{itemize}

%\begin{rSubsection}{Journal Reviews}{June 2016}{Site Engineer}{}
%\item On-site internship under this leading construction company. Learned and implemented various aspects such as quantity estimation, labour management and safety precautions.
%\end{rSubsection}


%\end{rSection}


%	EXAMPLE SECTION
%----------------------------------------------------------------------------------------


%\newpage



\end{document}
